\documentclass[11pt,a4paper]{article}

\usepackage{amsmath,amssymb}
\usepackage{graphicx}
\usepackage{epsfig}
\usepackage{dirtytalk}
\usepackage{dsfont}

\begin{document}

\title{IMRaD style}
\author{Evan Huynh, Mantas Rakstys and Hassan}

\maketitle

\begin{abstract}
 This report consists of the explanation for four different problems in the project for our group.
\end{abstract}

\tableofcontents

\section{Introduction}

\section{Results}
\section{Problem 1}
Short summarization: How many regions can \(n\) lines creates, assumes that there is no parallel nor concurrency.

Proposed solution: With the given condition, any new drawn line must cross all existing ones at \(n\) points. That means the line must also cross \(n\) regions enclosed by a given number of lines since they are all line up. 

Equation found: Let a be the number of regions created by \(n\) lines. We know that:
\[a_{n} = a_{n-1} + n + 1, \ n>1, \ n \in \mathds{Z} \]

Solving by Mathematica gives us: 
\

\section{Discussion}

\begin{thebibliography}{99}
 
\end{thebibliography}

\end{document}
